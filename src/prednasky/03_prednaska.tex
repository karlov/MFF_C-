\section{\texorpdfstring{Hierarchie pameti}{Hierarchie pameti}}
\vspace{5mm}
\large

\begin{definition}
	Dynamická paměť - jeden kondensátor.

	Každé čtení je destruktivní, takže čtení musí zároveň zapsat.
	Musíme taky obnovovat data dejme tomu každé 100ms.
\end{definition}

Paměť je taky omezena kvůli rychlosti světla, paměť je moc daleko od CPU.
Pokud máme 1Gz procesor, tak během jednoho taktu světlo uletí 30cm.

Typy cache
\begin{enumerate}
	\item Plně asociativní cache. Zavedeme cache line (bloky).
		V praxe se nepoužívá, protože potřebujeme hodně HW pro porovnávaní a shromažďovaní dat.
	\item Přímo mapovaná cache

		Nevýhody: (cache aliasing) viz uvodní příklad násobení matic při velikosti řádku $2^i$.
	\item Kvůli nevýhodám 1, 2 používá se kompromis - Množinově asociativní cache.
		Plně asociativní uvnitř podmnožin.
	\item Dalším vylepšením je hierarchie cachi. Aby cache byla rychlá, musí být blízko procesoru, takže musí být malá. Proto první cache je malá, druhá je větší.
\end{enumerate}

Dotaz: menší instrukční sada, znamená lepší cachovani v $L1I$.

\subsection{Memory Management Unit}

\begin{note}
Pozor $-O3$ optimalizaci neni rozumne zapinat gobalně, ale jenom lokalně kde očekáváme, že může vyrazně zlepšit.
\end{note}

\subsection{Cviko}

\begin{example}
	Při překladu prohozeni pole překladač nepočitá 2. index protože si pohopil, že procházíme pole sekvenčně.
	Takže si pořidil 2 pointry které sinchronně incrementuje.
\end{example}

\begin{example}
	V přikladu sqdiff najdeme bitový trik pro absolutní hodnotu.
\end{example}

\begin{example}
	Najdeme zneužití instrukce \textbf{adcl} pro podminěné přičtení jedničky.
\end{example}

\begin{example}
	Vypočet Fibonacci. Všimneme si, že -O3 nešahá do pamětí, všechno počita v registrech.

	Pak místo +4 pro další prvek v pole, si pořidíme další pointer který ukazuje na 2. prvek v poli.
\end{example}

\begin{example}
	V přikladu vypočtu polynomu při $-O3$ nejsou žadne cykly. Proč?

	Protože kompilátor si všiml, že pole má pevnou velikost, proto cyklus byl uplně rozbalen.
\end{example}
