\section{\texorpdfstring{Hardware architecture}{Hardware architecture}}
\vspace{5mm}
\large

\subsection{Historický vývoj architektur}
Historicky procesor 8086 $\to$ x86 $\to$ AMD64.

U x86 nás zajímá 32b mod s lineární adresaci.
DOS by fungoval na stejném procesoru trochu jinak.

\subsection{x86}

\paragraph{Datové typy}
\begin{itemize}
	\item integer 8/16/32 a nějaké operace s 64b integery.
	\item float\\
		a) 32 (24 + 8) - float\\
		b) 64 (53 + 11) - double\\
		c) 80 (64 + 16) - long double\\
		d) jen občas 16 (11 + 5) pouze ve vektorech.
		Používá se v grafice
	\item vektory 128b (homogenní, všechno kromě long double jako základní typ)
\end{itemize}

\paragraph{Paměť}
Z pohledu programu pole bytů. Von Neumann architektura, společná pro kod a data.
Ne všechna častí paměti lze používat. Když šáhneme na zakázanou část, tak OS zabije.

Přistup na špatně zarovnaný vektor selže!

\paragraph{Registry}
Univerzální registry, lze dělat skoro cokoliv.
\begin{itemize}
	\item EAX\\
		Původně na 8086 byl registr AX, spodní část AL, horní AH.\\
		Pak přidali ještě 32 bitů.
	\item EBX
	\item ECX
	\item EDX
\end{itemize}

Registry co složili pro cílovou adresu.
Dnes jsou univerzální, ale na rozdíl od EAX nelze použit dolní častí.
\begin{itemize}
	\item ESI - source index
	\item EDI - destination index
\end{itemize}

\begin{itemize}
	\item EBP\\
		BP - base pointer. Ukazuje dle konvence na oblast v paměti kde jsou lokální proměnné a parametry. Dnes už je univerzálním.
	\item ESP\\
		SP - stack pointer. ESP ukazuje na poslední adresu v Stacku.
	\item EPP - instruction pointer.
	\item EFLAGS - každé bity mají jiný význam.\\
		Např zero - poslední aritmetická operace byla 0.\\
		Přetečení (znaménkové a bezznaménkové)\\
		Řidiší bity, např D - direction.\\
\end{itemize}

Floutové registry
\begin{itemize}
	\item ST0 - ST7, tvoří float stack. Dnes se moc nepoužívá
	\item FPUSR (FPU status registr) - jako flagy.
	\item FPUCR (FPU control registr) - řídí jestli při dělení 0 mu vyskytnout výjimka apod.
\end{itemize}

Registry vektorové jednotky
\begin{itemize}
	\item XMM0 - XMM7 128 bitové
	\item YMM0 - YMM7 256 bitové
	\item MXCSR (control and status) - další flagy.
\end{itemize}

\begin{note}
	Lze oddělovat Int, Float, Vector část procesoru.
\end{note}

\paragraph{Assembler}
Intel (windows) vs AT\&T (linux).

Např přičteni 1 k registru EBX

\begin{itemize}
	\item Intel: ADD EBX, 1
	\item  AT\&T addl \$1, \%ebx \\
		kde l znamená (long), q by bylo 64 atd.\\
		před literálem je dolar, jinak by jednička znamenala místo v paměti.
\end{itemize}

Např přičteni 1 k místu v paměti uložené registru EBX
\begin{itemize}
	\item Intel: ADD DWORD PTR [EBX], 1
	\item  AT\&T addl \$1, (\%ebx)
\end{itemize}

\begin{agreement}
	Jelikož jsme na Linuxu, budeme používat AT\&T.
\end{agreement}

Operandy instrukce

\begin{itemize}
	\item \$ číslo
	\item \%registr
	\item číslo - adresa v paměti
	\item (\%reg) - adresace registrem
	\item číslo (\%registr) - adresa = reg + číslo
	\item číslo (\%r1, \%r2, mul) - $adresa = r1 + mult * r2 + číslo$ \\
		mul je 1,2,4,8.\\
		hlavně se hodí u adresaci poli.
\end{itemize}

Nechceme psát v assembleru, ale rozumět co vygeneroval překladač.
Pokud je neefektivní, chceme mu pomoct kod zlepšit.

\paragraph{ABI}
ABI - application binary interface, specifické pro architekturu, OS a jazyk (C).
Ostatní jazyky umí linkovat C-kovou knihovnu.

Řiká např kde je program v paměti. Ale nás toto nezajímá.

Např volací konvence.
Bylo by dobrý předávat parametry v registrech. Toto obecně nejde, protože registrů je málo a můžou se použit pro jiné veci.
Proto je dohodnutí že volající přidává parametry na stack. Dolu je první argument.
Je potřeba pro funkce s proměnlivým počtu argumentů.

Někde na zásobníku je návratová adresa.
Nahoře od ni je na starosti Volajícího. Pod ní Volaný.

\begin{definition}
	Stack frame -
\end{definition}

Výsledek funkce se vrátí buď v EAX nebo EAX/ADX (pokud větší než 64).
Float v ST0.
Pokud struktura, tak výjimka, volaný uloží v paměti. Posílá adresu v paměti.

\paragraph{Scratch} Věci co může měnit volaný:
\begin{itemize}
	\item EAX, ECX, EDX
	\item ST0-ST7
	\item část EFLAGS.
\end{itemize}

Zbytek musí zůstat jako dříve.

EBX není scratch, protože se používá pro sdílené knihovny, musí pracovat s relativní paměti (nahrávají se kamkoliv).
EBX ukazuje kde jsou data sdílené knihovny.

\subsection{AMD64}
Změny oproti x86:
\begin{itemize}
	\item EAX se rozšíří na horních 32 bitů $\to$ RAX.\\
		Analogicky pro EBX, ECX, EDX, ESI, EDI, EBP, ESP.\\
		Jsou univerzální proto se jmenuji R0-R7.\\
		Pozor, 16 bitové instrukce jsou pomalé.
	\item Nově máme R8-R15.
	\item FPU se moc nezměnila, jen veškeré počítaní jsou vektorové.
		Pozor nemám 80 bitový float, na toto se stále používá x86.
	\item Nově XMM8-15, YMM8-15.
\end{itemize}

Jelikož registrů je hodně, parametry se přidávají v registrech.
Pokud se nevešlo - tak na zásobníku.
Celkem 6 celočíselných argumentů v registrech, 8 float argumentů.


\subsection{Disassebler}

