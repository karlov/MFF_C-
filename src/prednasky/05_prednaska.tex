\section{\texorpdfstring{Programování na SMP: Procesy a vlákna}{Programování na SMP: Procesy a vlákna}}
\vspace{5mm}
\large

Jak programovat na SMP
\begin{itemize}
	\item Oddělené procesy: rozdělíme výpočet na paralelní kusy.
		Data sdílíme přes Unixove sockety, roury apod.

		Výpočet je ekvivalentní výpočtu na několika počítačích (třeba přes SSH).
		Nevyužíváme že máme SMP.
	\item Threads: kompletně sdílíme peměťový prostor.

		Pozn: v Linuxu skoro není rozdíl mezi Procesem a Vláknem.

		Vlákno ale mají vlastní IP a zásobník.

		Kontejner funguje tak, že sdílí jen málo namespace (třeba nesdílí síťový).

	\item Kompromis mezi dvěma přístupy nahoře: explicitní sdílení (např blok sdílené pamětí v Linuxu).
\end{itemize}

\begin{definition}
	Unixovy signal - softwarová přerušení (procesu se posílá malé číslo). Program se zastaví, a vykoná daný signal.
\end{definition}

Porovnaní

Problémy se sdílení
\begin{itemize}
	\item synchronizace
		\begin{itemize}
		\item race conditions
		\item relativní viditelnost změn
		\end{itemize}
	\item efektivita
		\begin{itemize}
			\item MESSI když musí posílat cache line mezi procesory.
			\item Ideálně chceme minimalizovat počet sdílených R/W data.
			\item Falešné sdílení: read only a sdílená data ve stejné cache lině \\
				Lze řešit zarovnáním, viz
		\end{itemize}
	\item bezpečnost syscallů a knihovní funkce (viz dokumentace funkce jestli lze použit paralelně).
		Rozlišuje se toto:
		\begin{itemize}
			\item Nebezpečné
			\item thread-safe
			\item signal-safe
			\item specifické záruky
		\end{itemize}
\end{itemize}

\begin{example}
	Příkladem specifické záruky je log.
	Různá vlákna chtějí psát do logu.

	Posouvaní ukázatelu na konec souboru (viz mode O\_APPEND) je atomické, takže nepotřebujeme zamykat.
\end{example}

Nechceme sdílet vůbec, takže třeba budeme redundantně kopírovat instanci pro každé vlákno.
Pak sečteme po práci všech vláken.
Toto vede na \emph{thread-local variables}.

\subsection{Thread-local variables}

\begin{itemize}
	\item pthread\_getspecific - otravné a pomalé
	\item GCC: \_\_thread
	\item C11: \_Thread\_local (s použitím hlavičkového souboru, přejmenuje na thread\_local).
\end{itemize}

\begin{note}
	\_Thread\_local se na Linuxu implementuje pomoci tzv segmentovaných registrů (historická featura).

	Přičítá se offset segmentu k adrese.
\end{note}

\subsection{Synchronizační primitiva}

Hlavně různé zámky.
\begin{definition}
	Synchronizační primitiva zaručuje že v daný okamžik se s daty pracuje pouze jediné vlákno.
\end{definition}

\begin{definition}
	Mutex - mutual exclusion. Má stavy zamčeno/odemčeno.

	Pokud voláme Lock(Acquire), čekáme až vlastník ho odemkne.
	Pokud voláme Unlock(Release), tak na nic nečekáme.
\end{definition}

\begin{definition}
	Semafor - stav je přirozené číslo.
	Operace: Down, Up (snižení a zvyšení o 1).

	Pokud nula s voláme down, čekáme pokud se zvýší na 1.

	Pozn: operace se jmenuje všelijak různě.
\end{definition}

\begin{definition}
	Condition variable - u mutexu nemůžeme počkat, až DS bude v nějakém stavu.

	Stav: mutex + fronta čekajících procesu.

	Operace: Wait(čekaní na změnu stavu), Signal.

	Pak ručně kontrolujeme v jakém stavu je struktura, pokud stav je jiný - voláme Wait znovu.
	Aby to bylo atomické, máme Mutex.

	Příklad čekaní:
	\begin{alltt}
		mutex.Lock();
		while(stav != wanted)
		\tab condVar.Wait();
		mutex.Unlock();
	\end{alltt}

	Příklad update:
	\begin{alltt}
		mutex.Lock();
		stav = wanted;
		mutex.Unlock();
		condVar.Signal;
	\end{alltt}
\end{definition}

\begin{example}
	Fronta

	Použijeme Mutex M(chraní vnitřní stav, aby nevznikal race condition pro vnitřní stav), Semafor S(počítá prvky).

	\begin{alltt}
		void Enqueve(x)
		\{
		\tab M.Lock();
		\tab Add(x);
		\tab M.Unlock();
		\tab S.Up
		\}
	\end{alltt}

	\begin{alltt}
		void Dequeve(x)
		\{
		\tab S.Down
		\tab M.Lock();
		\tab Remove(x);
		\tab M.Unlock();
		\}
	\end{alltt}

	Mutex je korektní: nevzníkné deadlock, uvnitř Mutexu nic nezamykáme.
	Co ale Semafor? Co když po $S.Down$ nikdo jiný odstraní prvek dřív než já?

	Invariant (rezervace na prvky): \# prvků v seznamu $=$ semafor + \# rezervaci.
	Platí když není zamčený Mutex.
\end{example}

\subsubsection{Rychlost}
Kdysi každá operace s synchronizační primitivy potřebuje Kernel volaní (potřebujeme uspívat vlákna).

Na Linuxu pokud nehrozí race condition, tak všechno se odehrává v user space (futex).
Pokud ale může race condition, jdeme do Kernelu.

\begin{note}
	Bache, zámek způsobí sdílena R/W mezí procesory (sběrnice).
	Takže implementace sama o sobě není pomalá, ale kvůli pamětí jo.
\end{note}

\begin{definition}
	Granularita: zámek na každý prvek nebo na celou DS obecně.

	Existuje ale kompromis: hash zámky.
	Musíme davat zámky do různých cache line.
\end{definition}

\subsection{Race Conditions}

\begin{definition}
	Deadlock
\end{definition}


\section{\texorpdfstring{Cvičení SMP}{Cvičení SMP}}
\vspace{5mm}
\large

\begin{example}
	Paralelizace přes xargs
	\begin{alltt}
		xargs -n
	\end{alltt}

\end{example}

\begin{example}
	fork: kopirovaní procesu.
	Předpřipravíme DS (init), pak jen kopirujeme proces.

	Kdyby jen kopírovaní, tak hodně pomalé. Použijeme trick copyon write mechanismus.

	Data jsou readonly, při zapisu kernelové přirušení, stranka se nakopiruje do lokalní kopie.
\end{example}

\begin{note}
	Fork vrací 2x, aby šlo odlišit kde jsme.
	\begin{alltt}
		int main()\{
		\tab if(fork() == 0) // potomek
		\tab else \\rodič, dostane pid potomka
		\}
	\end{alltt}

\end{note}
Pomocné funkce
\begin{itemize}
	\item \textbf{waitpid} - čeká na process.
	\item \textbf{\_exit} vracení z potomka.
\end{itemize}

\begin{example}
	NGINX používá všude fork, protože když spadne jeden worker, tak master ho může předat dalšímu. Ostatní workery to neovlivní.
\end{example}

\subsection{Pthreads}

Proč $void*$, protože typovací systém C. Chceme ale moct přidat cokoliv.

Funkce:
\begin{itemize}
	\item \textbf{pthread\_create} - vytváří vlákna.
	\item \textbf{pthread\_join} - čeká až vlákna doběhnou.
\end{itemize}

pthreads\_rwlock
neomezený read, ale pokud někdo zapisuje - ostatní nemohou číst.

\subsection{Atomické operace}

GCC(na nativním typu pro procesor, typicky Long)
\begin{itemize}
	\item \_\_sync\_fetch\_and\_add - atomický ++\\
		podobně XOR, SUB, AND

		Pokud procesor neumí, tak se asi nepřeloží.
	\item \_\_sync\_compare\_and\_swap - atomický swap\\
		Separátně pro bool a int.
	\begin{alltt}
		int compare_addex(\& addr, oldval, newval)\{
		\tab if(*addr == oldval)
		\tab \tab *addr = newval;
		\tab \}
	\end{alltt}
	pokud vrátí \textbf{oldval} tak se změna povedla, jinak nepovedla.
\end{itemize}

\begin{exercise}
	Počet dělitelů čísla.
\end{exercise}

\begin{exercise}
	Faktorizace.
\end{exercise}

\begin{exercise}
	Pomoci atomických operaci lock-free spojak.
	Pro jednoduchost jednosměrný a umí
	\begin{itemize}
		\item insert (začátek nebo konec?)
		\item můžeme jednoduše udělat Delete?
	\end{itemize}

	Procházíme do konce, pak použijeme atomický swap.
	Nebo taky na začátek, pak nemusíme aktualizovat tail pointer.
\end{exercise}

\begin{note}
	Pokud měříme čas běhu vice vláknového programu, tak to dá násobek reálného času.
	Můžeme změřit čas strávený každým vláknem pomoci $clock\_gettime$.
\end{note}
