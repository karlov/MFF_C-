\section{\texorpdfstring{Uvod, Cko}{Uvod, Cko}}
\vspace{5mm}
\large

%\begin{definition}
%Matice sousednosti grafu G
%\end{definition}

\subsection{Uvod}

\begin{itemize}
	\item V tomto předmětu se zaměříme na PC hw architekturu.
	\item Cílovým OS je Linux, jelikož běží na většině super počítačů, klastrů apod.
		Ostatní OS stavěné na výkon se chovají podobně.
	\item Cko jako cílový jazyk (bohužel občas místo kompilátoru je komplikator).
		GCC jako cílový kompilátor.
\end{itemize}

\begin{example}
	Motivační příklad: Transpozice matic.
	//TODO obrázek

	a) Křivka naivního algoritmu roste (proč? na RAMu by neměla) + spiky.
	\begin{itemize}
		\item Na začátku skoky lze vysvětlit velkosti $L3$ cache, ale ne všude.
			Hlavně spacial lokality v hlavní paměti.
		\item skoku u $2^i$ protože cache není plně asociativní.
			Z množiny čísel se stejnými spodní bity, jenom málo se může dostat to cache na stejné místo.
			Podobně u dělitelů mocnin 2ky.
	\end{itemize}
	b) Je to naivní algoritmus + snažíme se vyhnout mocninám 2ky tak, že přidáváme konstantní mezery mezi řádky.
	Znázorňuje zelená křivka.
	Proč roste jako exp na nejmenších velikostech? Vysvětlíme později.

	c) Tiling algoritmus \\
	d) Transpose and swap (cache oblivious algoritmus).
	Chová se o něco lip než tiling.

\end{example}

\begin{note}
	Dotaz: má smysl při inicializaci algoritmu zjistit velikost cache?

	Obecná odpověď je spíše NE, ze 2 důvodů:
	\begin{enumerate}
		\item Reálná hierarchie může být extra složitá. Jenom velikost nepomůže.
		\item U Transpose and Swap větší role hraje režie rekurze.
			Protože rekurze stoji víc než přistup do paměti.
			Takže se musíme zastavit na takové největší velikosti podproblému, který se vejde do cachi.
			Pokud víme, že $L_1 > 16kb$, tak funguje matice $64x64$.
		\item Nastavení parametrů může škodit kompilaci.
			Jde ale obejit tak, že zkompilujeme několik verzi algoritmu s různé parametry, pak rychlé zjistit který je rychlejší a nadále používat.

			Např Linux kernel má k dispozici 4 algoritmy pro sw REID. Při bootu zkontroluje který je rychlejší a bude používat po celou dobu.
	\end{enumerate}
\end{note}

\subsection{Cko}
Tady budou pouze komentáře k slajdům \href{http://mj.ucw.cz/papers/cecko.pdf}{Uvod do Cka}.

\paragraph{Vývoj dialektů Cka}
\begin{enumerate}
	\item K\&R Cko, původní návrh popsaný v knize.
	\item První standardizace, ANSI a ISO.
		Je známá jako C89. Z ní se vytvořili dialekty dle překladače a taky C++.
	\item C99 nejlepší featury z dialektů.
		Pak vznikla ještě C11 verze a C17 (druhá jen opravovala chyby první).
	\item Občas Cko přebralo featury zpět od C++, např v C11 memory management pro paralelní programovaní.
\end{enumerate}

\paragraph{Číselné typy}
\begin{enumerate}
	\item Není specifikováno, jestli \textbf{char} je \emph{signed} nebo \emph{unsigned}.
	\item Typ \textbf{bool} máme protože kompilátor lepe optimalizuje.
		Taky lepší semantický význam.
		Jmenuje se ale \textbf{\_Bool} aby kompilace nerozbila stávající programy, které již měli bool přes \emph{typedef}.
		Pokud ale chci pravě \textbf{bool}, stačí použit $include <stdbool.h>$.
	\item stejně jako v C++, existuji typy pevné délky, např u\_int32.
		Hlavně se používají u síťových protokolu, taky inline assembler.
\end{enumerate}

Union má pouze jednu ze dvou položek v pamětí, sám neví který to je. Je potřeba odvodit z kontextu.

\paragraph{Částečné typy}
\begin{enumerate}
	\item struct who\_knowns\_what *
	\item int f() - nevíme parametry
	\item int f(void) - nemá žádný parametr
	\item int [] - pole neznámé velikosti
\end{enumerate}

\paragraph{Reprezentace v pamětí}
\begin{enumerate}
	\item pozor na bitová pole při paralelním programovaní. (necelé bytů se hlavně používají hlavně pro síťové pakety)
	\item pozor 2kový doplněk není zaručený.
	\item C11 má operátor \_AlignOf(type), nejčastěji velikost typu.
	\item překladač nesmí prohodit pořadí položek uvnitř.
		Protože se zaručuje, že 2 struktury zděděné ze stejného typů mají stejný prefix v pamětí.
\end{enumerate}

\paragraph{Literály}
\begin{enumerate}
	\item Pozor, přidaní nuly na začátek změní literál na 8 soustavu.
		Změní se i hodnota.
	\item Veškeré celočíselné počítaní se provádí v typu int.
	\item Floating-point literály bez ztráty přesností se zapisuji s exponentem, například 0x1.ffep10, kde p je oddělovač pro exponent.
\end{enumerate}

\paragraph{Operátory}
\begin{itemize}
	\item používáme pointer->item, protože při zápisu *pointer.item tečka má větší prioritu, museli bychom psát (*pointer).item
	\item $a = x++$ pak $a$ má původní hodnotu. $a = ++x$ pak $a$ má hodnotu po inkrementu.
		$x = x++$ je hloupost, není definované.
	\item Hádanka $a+++++a = $ ??
	\item nepoužívat bitové operace u znaménkových typů
	\item $!!x$ je přetypovaní na bool
	\item nemodulit záporným číslem
	\item float není lineárně uspořádané. $NaN \ne NaN$.
	\item počítaní s unsigned a signed dává unsigned
	\item nepsat $printf("\%d \%d \%d", a++, a++, a++$.
		Dobrým zvykem je měnit hodnotu proměnné pouze jednou v příkazu.
\end{itemize}

Nedefinované chovaní - cokoliv se může stát.

\paragraph{Pointry}
\begin{itemize}
	\item pointer na pole je něco jiného než pointer na první prvek. S pointrem na pole se ale běžně nepracuje.
	\item pointer na fce se zapisuje $int (*f)(int a);$
	\item nepoužívat složitější konstrukce jako pointer na pointer poli pointrů.
\end{itemize}

\paragraph{Modifikátory}
\begin{itemize}
	\item u \emph{const} záleží kde je. Protože $const int *x$ je konstantní integer, když $int * const x$ je const pointer.
	\item \emph{register} int x - je historický, hint překladači, že může uložit do registru místo pamětí.
		Pak ale nejde použit pointer. Dneska se nepoužívá, překladač optimalizuje lepe.
	\item \emph{volatile} (hist že se může nečekaně změnit). Třeba namapovaná periferie do pamětí.
		Třeba nějaký port cpe data, pokud nenapíšeme volatile, tak překladač může vyoptimalizovat 2 čtení na 1.
		Lze taky použit pro zrušení optimalizace pro měření.
	\item \emph{restrict} - na adresu lze přistupovat pouze tímto pointrem nebo odvozeným.
		Pak překladač lepe optimalizuje. Např lze použit při kopirovaní pamětí, zaručí že 2 useky se nepřekrývají.
		Používá to funkce memcp nebo memmove.
	\item static int x - u globální jako private. Je viditelná uvnitř modulu.
	\item extern int x - společná proměnná pro několik modulu, stejný název.
	\item inline - místo volaní se vloží. Dneska překladač um sám.
\end{itemize}
