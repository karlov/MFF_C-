\section{\texorpdfstring{Uvod, Cko}{Uvod, Cko}}
\vspace{5mm}
\large

\subsection{Uvod}

\begin{itemize}
	\item V tomto předmětu se zaměříme na PC hw architekturu.
	\item Cílovým OS je Linux, jelikož běží na většině super počítačů, klastrů apod.
		Ostatní OS stavěné na výkon se chovají podobně.
	\item Cko jako cílový jazyk (bohužel občas místo kompilátoru je komplikator).
		GCC jako cílový kompilátor.
\end{itemize}

\begin{example}
	Motivační příklad: Transpozice matic.
	//TODO obrázek

	a) Křivka naivního algoritmu roste (proč? na RAMu by neměla) + spiky.
	\begin{itemize}
		\item Na začátku skoky lze vysvětlit velkosti $L3$ cache, ale ne všude.
			Hlavně spacial lokality v hlavní paměti.
		\item skoku u $2^i$ protože cache není plně asociativní.
			Z množiny čísel se stejnými spodní bity, jenom málo se může dostat to cache na stejné místo.
			Podobně u dělitelů mocnin 2ky.
	\end{itemize}
	b) Je to naivní algoritmus + snažíme se vyhnout mocninám 2ky tak, že přidáváme konstantní mezery mezi řádky.
	Znázorňuje zelená křivka.
	Proč roste jako exp na nejmenších velikostech? Vysvětlíme později.

	c) Tiling algoritmus \\
	d) Transpose and swap (cache oblivious algoritmus).
	Chová se o něco lip než tiling.

\end{example}

\begin{note}
	Dotaz: má smysl při inicializaci algoritmu zjistit velikost cache?

	Obecná odpověď je spíše NE, ze 2 důvodů:
	\begin{enumerate}
		\item Reálná hierarchie může být extra složitá. Jenom velikost nepomůže.
		\item U Transpose and Swap větší role hraje režie rekurze.
			Protože rekurze stoji víc než přistup do paměti.
			Takže se musíme zastavit na takové největší velikosti podproblému, který se vejde do cachi.
			Pokud víme, že $L_1 > 16kb$, tak funguje matice $64x64$.
		\item Nastavení parametrů může škodit kompilaci.
			Jde ale obejit tak, že zkompilujeme několik verzi algoritmu s různé parametry, pak rychlé zjistit který je rychlejší a nadále používat.

			Např Linux kernel má k dispozici 4 algoritmy pro sw REID. Při bootu zkontroluje který je rychlejší a bude používat po celou dobu.
	\end{enumerate}
\end{note}

\subsection{Cko}
Tady budou pouze komentáře k slajdům \href{http://mj.ucw.cz/papers/cecko.pdf}{Uvod do Cka}.

\paragraph{Vývoj dialektů Cka}
\begin{enumerate}
	\item K\&R Cko, původní návrh popsaný v knize.
	\item První standardizace, ANSI a ISO.
		Je známá jako C89. Z ní se vytvořili dialekty dle překladače a taky C++.
	\item C99 nejlepší featury z dialektů.
		Pak vznikla ještě C11 verze a C17 (druhá jen opravovala chyby první).
	\item Občas Cko přebralo featury zpět od C++, např v C11 memory management pro paralelní programovaní.
\end{enumerate}

\paragraph{Číselné typy}
\begin{enumerate}
	\item Není specifikováno, jestli \textbf{char} je \emph{signed} nebo \emph{unsigned}.
	\item Typ \textbf{bool} máme protože kompilátor lepe optimalizuje.
		Taky lepší semantický význam.
		Jmenuje se ale \textbf{\_Bool} aby kompilace nerozbila stávající programy, které již měli bool přes \emph{typedef}.
		Pokud ale chci pravě \textbf{bool}, stačí použit $include <stdbool.h>$.
	\item stejně jako v C++, existuji typy pevné délky, např u\_int32.
		Hlavně se používají u síťových protokolu, taky inline assembler.
\end{enumerate}

Union má pouze jednu ze dvou položek v pamětí, sám neví který to je. Je potřeba odvodit z kontextu.

\paragraph{Částečné typy}
\begin{enumerate}
	\item struct who\_knowns\_what *
	\item int f() - nevíme parametry
	\item int f(void) - nemá žádný parametr
	\item int [] - pole neznámé velikosti
\end{enumerate}

\paragraph{Reprezentace v pamětí}
\begin{enumerate}
	\item pozor na bitová pole při paralelním programovaní. (necelé bytů se hlavně používají hlavně pro síťové pakety)
	\item pozor 2kový doplněk není zaručený.
	\item C11 má operátor \_AlignOf(type), nejčastěji velikost typu.
	\item překladač nesmí prohodit pořadí položek uvnitř.
		Protože se zaručuje, že 2 struktury zděděné ze stejného typů mají stejný prefix v pamětí.
\end{enumerate}

\paragraph{Literály}
\begin{enumerate}
	\item Pozor, přidaní nuly na začátek změní literál na 8 soustavu.
		Změní se i hodnota.
	\item Veškeré celočíselné počítaní se provádí v typu int.
	\item Floating-point literály bez ztráty přesností se zapisuji s exponentem, například 0x1.ffep10, kde p je oddělovač pro exponent.
\end{enumerate}

\paragraph{Operátory}
\begin{itemize}
	\item používáme pointer->item, protože při zápisu *pointer.item tečka má větší prioritu, museli bychom psát (*pointer).item
	\item $a = x++$ pak $a$ má původní hodnotu. $a = ++x$ pak $a$ má hodnotu po inkrementu.
		$x = x++$ je hloupost, není definované.
	\item Hádanka $a+++++a = $ ??
	\item nepoužívat bitové operace u znaménkových typů
	\item $!!x$ je přetypovaní na bool
	\item nemodulit záporným číslem
	\item float není lineárně uspořádané. $NaN \ne NaN$.
	\item počítaní s unsigned a signed dává unsigned
	\item nepsat $printf("\%d \%d \%d", a++, a++, a++)$.
		Dobrým zvykem je měnit hodnotu proměnné pouze jednou v příkazu.
\end{itemize}

Nedefinované chovaní - cokoliv se může stát.

\paragraph{Pointry}
\begin{itemize}
	\item pointer na pole je něco jiného než pointer na první prvek. S pointrem na pole se ale běžně nepracuje.
	\item pointer na fce se zapisuje $int (*f)(int\ a);$
	\item nepoužívat složitější konstrukce jako pointer na pointer poli pointrů.
\end{itemize}

\paragraph{Modifikátory}
\begin{itemize}
	\item u \emph{const} záleží kde je. Protože

	\begin{alltt}
		const int *x
	\end{alltt}
		je konstantní integer, když
	\begin{alltt}
		int * const x
	\end{alltt}
		je const pointer.
	\item \emph{register} int x - je historický, hint překladači, že může uložit do registru místo pamětí.
		Pak ale nejde použit pointer. Dneska se nepoužívá, překladač optimalizuje lepe.
	\item \emph{volatile} (hist že se může nečekaně změnit). Třeba namapovaná periferie do pamětí.
		Třeba nějaký port cpe data, pokud nenapíšeme volatile, tak překladač může vyoptimalizovat 2 čtení na 1.
		Lze taky použit pro zrušení optimalizace pro měření.
	\item \emph{restrict} - na adresu lze přistupovat pouze tímto pointrem nebo odvozeným.
		Pak překladač lepe optimalizuje. Např lze použit při kopirovaní pamětí, zaručí že 2 useky se nepřekrývají.
		Používá to funkce \emph{memcp} nebo \emph{memmove}.
	\item static int x - u globální jako private. Je viditelná uvnitř modulu.
	\item extern int x - společná proměnná pro několik modulu, stejný název.
	\item inline - místo volaní se vloží. Dneska překladač um sám.
\end{itemize}

\paragraph{Příkazy}

\begin{itemize}
	\item prázdný příkaz se může hodit. Ale je potřeba dávat pozor

	\begin{alltt}
		while(...);
	\end{alltt}
	zopakuje prázdný příkaz do nekonečna.

	Tady je ale potřeba napsat středníky, nebo \{\}
	\begin{alltt}
		if(...) a;
		else b;
	\end{alltt}

	\item výraz je příkaz (vyhodnotí se side efekty)
	\item for(a, b, c) cyklus je ekvivalentní tomuto
	\begin{alltt}
		a;
		while(b)
		\{
			\tab příkazy;
			\tab c;
		\}
	\end{alltt}

	\item příkaz goto je potřeba použit v následujících případech
		\begin{itemize}
			\item break ze 2 a vice cyklů.
			\item konečné automaty
		\end{itemize}
	\item jde zkombinovat while a switch
	\begin{alltt}
		switch()
		\{
		\tab while(b)
		\tab \{
		\tab 	\tab case 1:
		\tab 	\tab ;
		\tab \}
		\}
	\end{alltt}
	Hodí se třeba pro zpracovaní poli po 4.
	switch je pro $\mod 4$

	\item \_static\_assert kontrola v čase kompilace, že např integer je nějaké velikosti
\end{itemize}

\subsection{Preprocesor}
\begin{itemize}
	\item proměnlivý \# argumentů, makro $\langle$ stdarg.h $\rangle$.
		Taky makro va\_arg.
	\item trigrafy, používat nechceme, historická záležitost, hlavně nepsat omylem ?? char.
	\item \$include preprocesor nahrazuje kodem jiného souboru.
		Preprocesor pracuje na pp-tokenech
	\item gcc umí tzv linked optimisation - optimalizace mezi moduly.
\end{itemize}

\paragraph{Makra}
Jsou 3 druhy
\begin{enumerate}
	\item s závorky (má argumenty). Např
	\begin{alltt}
		\# define P1(x) x + 1
	\end{alltt}

	\item s závorky (bez argumentů)
	\begin{alltt}
		\# define Z() null.
	\end{alltt}
	\item bez závorky (závorky nebudou části expanze)
	\begin{alltt}
		\# define X x.
	\end{alltt}
\end{enumerate}

\begin{properties} Makra
	\begin{itemize}
		\item je zakázáno definovat stejné makro 2x.
			Undef ale jde provést libovolně krát.
			Proto v souborech často guardy.
		\item dle konvence $\langle header \rangle$ hledá mezi systémové soubory.
			$"header"$ - soubory definované programátorem.
		\item \# se používá při generovaní kodu třeba pro vypsaní chybové hlášky aby odkazovalo na původní soubor.
		\item \#pragma jsou specifické pro kompilátor.
		\item při aritmetice v makrech je potřeba používat závorky.
	\end{itemize}
\end{properties}

\paragraph{Ostatní}
Nezařazené komentáře

\begin{itemize}
	\item size\_t určuje velikost objektů.
	\item od C11 máme vnořené (lokální) funkce. Pro implementace se používá tzv trampolína (místo v pamětí kde se ukládají data)
	\item kod pro defaultní hodnotu
	\begin{alltt}
		a?:b = a ? a : b
	\end{alltt}
	\item range v case
	\begin{alltt}
		case 1...5:
			;
	\end{alltt}
	\item pokud jeden case propadá v další je konvenci psat komentář, jinak warning od kompilátoru
	\begin{alltt}
		case 1:
			\tab ...
			\tab // fall through
		case 2:
	\end{alltt}
	\item \_\_auto = 5, stejně jako var v C\#.
\end{itemize}

\paragraph{GCC atributy}
\begin{itemize}
	\item hot (hodně optimalizovat, hot path)
	\item cold (málo optimalizovat, moc se nespouští)
	\item noinline (hlavně pro profilovaní)
	\item flatten - inline všechno co se volá uvnitř.
		Na co se hodí??
	\item packed - nepřidávat padding
\end{itemize}

\subsection{Bitové triky}
Bitové triky mimo jiné umožňuji vyhnout se podmíněným skokům, tzv branchless programovaní.
Což hodně pomáhá branch predictoru.
\begin{exercise}
	\begin{enumerate}
		\item bitová rotace
		\item násobení konstantou
		\item zaokrouhlení na nejbližší násobek 16
		\item test zda je číslo mocnina 2ky
		\item zaokrouhlit na nejbližší mocninu 2ky.
		\item zrcadlení: bitová i bytová
		\item abs hodnota
		\item signum
		\item parita, lichý nebo sudý \# jedniček
	\end{enumerate}
\end{exercise}
\begin{proof}
	1.
	\begin{alltt}
		(x >> k) | (x << (32 - k))
	\end{alltt}
	2. Rozepsat jako součet mocnin 2ky, pak
	\begin{alltt}
		(x << 4) + (x << 2) ...
	\end{alltt}
	3. Potřebujeme vynulovat spodní 4 bity
	\begin{alltt}
		x \& 1 ... 110000
	\end{alltt}
	Postup jak vyrobit masku
	\begin{alltt}
		~(1 << 4) - 1
	\end{alltt}

	Pro zaokrouhlení nahoru
	\begin{alltt}
		(x + 15)/16
	\end{alltt}
	uprostřed
	\begin{alltt}
		(x + 8)/16
	\end{alltt}

	4. Potřebujeme najít první jedničku vpravo
	\begin{alltt}
		x \& (~(x - 1)) == x
	\end{alltt}
	Pokud máme zaručený 2kový doplněk, tak
	\begin{alltt}
		x \& (x - 1) == 0
	\end{alltt}

	5. Máme v registru 0000x, kde x je posloupnost bitů začínající 1.
	Vyrobíme stejně dlouhou posloupnost jedniček.
	\begin{alltt}
		x -= 1; //abychom dostali neostrou nerovnost
		x |= x >> 16;
		x |= x >> 8;
		...
		x |= x >> 1;

		x += 1; //čímž dostaneme chtěnou mocninu 2ky
	\end{alltt}

	Pro dolní odhad na začátku přidáme
	\begin{alltt}
		x = (x >> 1) + 1;
	\end{alltt}

	6. Divide and Conquer
	\begin{alltt}
		(x >> 16) | (x << 16)
		((x \& 0xff00ff00) >> 8) | ((x \& 0x00ff00ff) << 8)
		...
	\end{alltt}

	7.
	\begin{alltt}
		y = x >> 31; (aritmetický posun)
		(x ^ y) - y
	\end{alltt}
	Pro kladná y je same nuly. Dostaneme x.
	Pro záporná
	\[ x \textasciicircum y = \sim x = (-x - 1)\]
	\[ (-x - 1) - y = (-x - 1) - 1 = -x \]

	8. První řešení
	\begin{alltt}
		(x > 0) - (x < 0)
	\end{alltt}
	Druhé
	\begin{alltt}
		((unsigned)x) >> 31
	\end{alltt}

	9. Divide and Conquer
	\begin{alltt}
		x ^= (x >> 32);
		x ^= (x >> 16);
		x ^= (x >> 8);
		x ^= (x >> 4);
		x ^= (x >> 1);
		// máme XOR všeho

		x \&= 1;
	\end{alltt}

	10. Swap přes XOR. Funguje protože XOR je involuce
	\begin{alltt}
		a ^= b;
		b ^= a;
		a ^= b;
	\end{alltt}

	11. Jednostranný spojak přes XOR.
	Místo pointru na next a prev uložíme
	\[ prev \textasciicircum next \]
	Posouváme se pomoci XORu.
\end{proof}
